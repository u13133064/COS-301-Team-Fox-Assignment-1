\documentclass[a4paper,12pt]{article}
\usepackage{blindtext}
\usepackage[utf8]{inputenc}

\begin{document}

\begin{titlepage}

\newcommand{\HRule}{\rule{\linewidth}{0.5mm}} % Defines a new command for the horizontal lines, change thickness here

\center % Center everything on the page
 
%----------------------------------------------------------------------------------------
%	HEADING SECTIONS
%----------------------------------------------------------------------------------------

\textsc{\LARGE University of Pretoria}\\[1.5cm] 
\textsc{\Large COS 301 - Software Engineering}\\[0.5cm] 
\textsc{\large Team Fox}\\[0.5cm] 

%----------------------------------------------------------------------------------------
%	TITLE SECTION
%----------------------------------------------------------------------------------------

\HRule \\[0.4cm]
{ \huge \bfseries Software Requirements Specification and Technology Neutral Process Design}\\[0.4cm] 
\HRule \\[1.5cm]
 
%----------------------------------------------------------------------------------------
%	AUTHOR SECTION
%----------------------------------------------------------------------------------------

\begin{minipage}{0.4\textwidth}
\begin{flushleft} \large
\emph{Author(s):}\\
Gian Paolo \textsc{Buffo} % Your name
\end{flushleft}
\end{minipage}
~
\begin{minipage}{0.4\textwidth}
\begin{flushright} \large
\emph{Student number(s):} \\
14446619 % Student Number
\end{flushright}
\end{minipage}\\[4cm]

%----------------------------------------------------------------------------------------
%	DATE SECTION
%----------------------------------------------------------------------------------------

{\large \today}\\[3cm] % Date, change the \today to a set date if you want to be precise

%----------------------------------------------------------------------------------------
%	LOGO SECTION
%----------------------------------------------------------------------------------------

%\includegraphics{Logo}\\[1cm] % Include a department/university logo - this will require the graphicx package
 
%----------------------------------------------------------------------------------------

\vfill % Fill the rest of the page with whitespace

\end{titlepage}

\newpage

\section{Introduction}

The requirements specification should ultimately contain sufficient information such that the
system could be largely developed by a third party without further input. To this end the
requirements must be precise and testable.
The requirements need not be fully specified up-front. One might start with the vision, scope
and architectural requirements, perform an upfront software architecture engineering phase and
then iteratively elicit the detailed requirements for a use case, build, test and deploy the use
case before adding the detailed requirements for the next use case. Such an approach follows
solid engineering phase for the core software infrastructure/architecture with an agile software
development approach within which the application functionality is developed iteratively.

\section{Vision}

A short discussion of the project vision, i.e. what the client is trying to achieve with the project
and the typical usage scenarios for the outputs of the project.

\section{Background}

A general discussion of what lead to the project including potentially
\begin{itemize}
\item business/research opportunities,
\item opportunities to simplify/improve some aspect of life/work or community,
\item problems your client is currently facing,
\item . . .
\end{itemize}

\section{Architecture Requirements}

The software architecture requirements include the access and integration requirements, quality
requirements and architectural constraints.

\subsection{Access Channel Requirements}

Specify the different access channels through which the system's services are to be accessed by
humans and by other systems (e.g. Mobile/Android application clients, Restful web services
clients, Browser clients, . . . ).

\subsection{Quality Requirements}

Specify and quantify each of the quality requirements which are relevant to the system. Examples of quality requirements include performance, reliability, scalability, security, 
flexibility,
maintainability, auditability/monitorability, integrability, cost, usability. Each of these quality
requirements need to be either quantified or at least be specified in a testable way.

\subsection{Integration Requirements}

This section specifies any integration requirements for any external systems. This may include
\begin{itemize}
\item the integration channel to be used,
\item the protocols to be used,
\item API specifications in the form of UML interfaces and/or technology-specific API specifications (e.g. WSDLs, CORBA IDLs, . . . ), and
\item any quality requirements for the integration itself (performance, scalability, reliability, security, auditability, . . . ).
\end{itemize}

\subsection{Architecture Constraints}

This specifies any constraints the client may specify on the system architecture include
\begin{itemize}
\item technologies which MUST be used,
\item architectural patterns/frameworks which must be used (e.g. layering, Services Oriented Architectures, . . . )
\item . . .
\end{itemize}

\section{Functional requirements and application design}
This section discusses the application functionality required by users (and other stakeholders).\\
\textbf{Additional information for this section is provided in the spec uploaded to the CS website.}

\subsection{Use case prioritization}

\subsection{Use case/Services contracts}

\subsection{Required functionality}

\subsection{Process specifications}

\subsection{Domain Model}

\section{Open Issues}

Discuss in this section
\begin{itemize}
\item any aspects of the requirements which still need to be specified,
\item around which clarification is still required, as well as
\item any discovered inconsistencies in the requirements.
\end{itemize}


\end{document}
