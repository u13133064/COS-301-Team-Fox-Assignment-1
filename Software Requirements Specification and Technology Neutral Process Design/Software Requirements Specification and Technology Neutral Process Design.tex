\documentclass[a4paper,12pt]{article}
\usepackage{blindtext}
\usepackage[utf8]{inputenc}

\begin{document}

\begin{titlepage}

\newcommand{\HRule}{\rule{\linewidth}{0.5mm}} % Defines a new command for the horizontal lines, change thickness here

\center % Center everything on the page
 
%----------------------------------------------------------------------------------------
%	HEADING SECTIONS
%----------------------------------------------------------------------------------------

\textsc{\LARGE University of Pretoria}\\[1.5cm] % Name of your university/college
\textsc{\Large COS 301 - Software Engineering}\\[0.5cm] % Major heading such as course name
\textsc{\large Team Fox}\\[0.5cm] % Minor heading such as course title

%----------------------------------------------------------------------------------------
%	TITLE SECTION
%----------------------------------------------------------------------------------------

\HRule \\[0.4cm]
{ \huge \bfseries Software Requirements Specification and Technology Neutral Process Design}\\[0.4cm] % Title of your document
\HRule \\[1.5cm]
 
%----------------------------------------------------------------------------------------
%	AUTHOR SECTION
%----------------------------------------------------------------------------------------

\begin{minipage}{0.4\textwidth}
\begin{flushleft} \large
\emph{Author(s):}\\
Gian Paolo \textsc{Buffo} % Your name
\end{flushleft}
\end{minipage}
~
\begin{minipage}{0.4\textwidth}
\begin{flushright} \large
\emph{Student number(s):} \\
14446619 % Student number
\end{flushright}
\end{minipage}\\[4cm]


%----------------------------------------------------------------------------------------
%	DATE SECTION
%----------------------------------------------------------------------------------------

{\large \today}\\[3cm] % Date, change the \today to a set date if you want to be precise

 
%----------------------------------------------------------------------------------------

\vfill % Fill the rest of the page with whitespace

\end{titlepage}

\newpage

%=====================================READ ME============================================
%----------------------------------------------------------------------------------------
% I have randomly assigned questions to all of us to get the ball rolling. Obviously,
% anyone can contribute to anything if they feel they have something to add/change.
% The names will be above each section.
% - GP
%----------------------------------------------------------------------------------------


%----------------------------------------------------------------------------------------
% To be done by: Jason
%----------------------------------------------------------------------------------------
\section{Introduction}

The requirements specification should ultimately contain sufficient information such that the
system could be largely developed by a third party without further input. To this end the
requirements must be precise and testable.
The requirements need not be fully specified up-front. One might start with the vision, scope
and architectural requirements, perform an upfront software architecture engineering phase and
then iteratively elicit the detailed requirements for a use case, build, test and deploy the use
case before adding the detailed requirements for the next use case. Such an approach follows
solid engineering phase for the core software infrastructure/architecture with an agile software
development approach within which the application functionality is developed iteratively.

%----------------------------------------------------------------------------------------
% To be done by: GP
%----------------------------------------------------------------------------------------
\section{Vision}

A short discussion of the project vision, i.e. what the client is trying to achieve with the project
and the typical usage scenarios for the outputs of the project.

%----------------------------------------------------------------------------------------
% To be done by: Jacques
%----------------------------------------------------------------------------------------
\section{Background}

A general discussion of what lead to the project including potentially
\begin{itemize}
\item business/research opportunities,
\item opportunities to simplify/improve some aspect of life/work or community,
\item problems your client is currently facing,
\item . . .
\end{itemize}

\section{Architecture Requirements}

The software architecture requirements include the access and integration requirements, quality
requirements and architectural constraints.

%----------------------------------------------------------------------------------------
% To be done by: Hlengekile
%----------------------------------------------------------------------------------------
\subsection{Access Channel Requirements}

Specify the different access channels through which the system's services are to be accessed by
humans and by other systems (e.g. Mobile/Android application clients, Restful web services
clients, Browser clients, . . . ).

%----------------------------------------------------------------------------------------
% To be done by: JoDan
%----------------------------------------------------------------------------------------
\subsection{Quality Requirements}

Specify and quantify each of the quality requirements which are relevant to the system. Examples of quality requirements include performance, reliability, scalability, security, 
flexibility,
maintainability, auditability/monitorability, integrability, cost, usability. Each of these quality
requirements need to be either quantified or at least be specified in a testable way.

%----------------------------------------------------------------------------------------
% To be done by: Jason
%----------------------------------------------------------------------------------------
\subsection{Integration Requirements}

This section specifies any integration requirements for any external systems. This may include
\begin{itemize}
\item the integration channel to be used,
\item the protocols to be used,
\item API specifications in the form of UML interfaces and/or technology-specific API specifications (e.g. WSDLs, CORBA IDLs, . . . ), and
\item any quality requirements for the integration itself (performance, scalability, reliability, security, auditability, . . . ).
\end{itemize}

%----------------------------------------------------------------------------------------
% To be done by: Jedd
%----------------------------------------------------------------------------------------
\subsection{Architecture Constraints}

This specifies any constraints the client may specify on the system architecture include
\begin{itemize}
\item technologies which MUST be used,
\item architectural patterns/frameworks which must be used (e.g. layering, Services Oriented Architectures, . . . )
\item . . .
\end{itemize}

\section{Functional requirements and application design}
This section discusses the application functionality required by users (and other stakeholders).\\
\textbf{Additional information for this section is provided in the spec uploaded to the CS website.}

%----------------------------------------------------------------------------------------
% To be done by: Sandile
%----------------------------------------------------------------------------------------
\subsection{Use case prioritization}
Critical
\begin{itemize}
\item Adding a conference paper
\item Adding an author to a conference paper
\item User being able to see all papers they have added or mentioned as authors to
\item Adding a researcher to a research group
\item Editing meta data 
\item Research leader being able to view all papers and their progress
\item The state of the paper (submitted, waiting, rejected, published)
\item Functionality to back up 
\item Add and remove authors anytime of the paper
\item Show history of papers
\item Staff members being able to access the portal

\end{itemize}
Important
\begin{itemize}
\item Head Of Department being able to view all papers
\item The sequence of authors(primary,second etc)
\item log everything 
\item Keep track of units, showing charts to see if they meet the target
\item Count units only when paper has been published
\item U.P. is the default of everything
\item Show the intend venue of paper and the type of the paper
\item Send a reminder of when the paper is due
\item unit paper appears by default once it has been stored
\item Search for an author
\item Head of department being able to view all units per stuff
\item An Administrator having complete access to the portal even on behalf of other users
\item Head of Department having complete access to the portal even on behalf of other users
\end{itemize}
Nice to have
\begin{itemize}
 \item profile of the researcher
 \item can adapt to a change such that it can be extended for other departments
 \item A user who's not an author adding a paper for someone else
\end{itemize}

%----------------------------------------------------------------------------------------
% To be done by: Kudzai
%----------------------------------------------------------------------------------------
\subsection{Use case/Services contracts}
Adding a conference paper
\begin{itemize}
    \item Preconditions
    \begin{itemize}
        \item User must be a staff member
        \item User must be logged in
        \item A paper must have at least one author
         \item Primary author must be specified
        \item User must enter the meta data about the paper
    \end{itemize}
    \item Postconditions
    \begin{itemize}
        \item Conference paper successfully added
    \end{itemize}
    \item Exceptions
    \begin{itemize}
        \item User can create a paper but does not have to be an author
    \end{itemize}
\end{itemize}
Adding an author to a conference paper
\begin{itemize}
    \item Preconditions
    \begin{itemize}
        \item User must be a staff member
        \item User must be logged in
        \item User can specify the number of co-authors
       
    \end{itemize}
    \item Postconditions
    \begin{itemize}
        \item Author would be added to a paper
    \end{itemize}
    \item Exceptions
    \begin{itemize}
        \item Author does not have to be a user
    \end{itemize}
\end{itemize}
User being able to see all papers they have added or mentioned as authors to one
\begin{itemize}
    \item Preconditions
    \begin{itemize}
        \item User must be logged in
        \item User must be an author or co-author to at least one paper
    \end{itemize}
    \item Postconditions
    \begin{itemize}
        \item User will be able to view their papers
    \end{itemize}
\end{itemize}
Adding a researcher to a research group
\begin{itemize}
    \item Preconditions
    \begin{itemize}
        \item User adding the researcher must be a researcher leader, Head of Department or an administrator
    \end{itemize}
    \item Postconditions
    \begin{itemize}
        \item Researcher added to research group
    \end{itemize}
\end{itemize}
The state of the paper (submitted, waiting, rejected, published)
\begin{itemize}
    \item Preconditions
    \begin{itemize}
        \item The paper must have already been added to the system
    \end{itemize}
    \item Postconditions
    \begin{itemize}
        \item The user can view the status of the paper
    \end{itemize}
\end{itemize}
Editing meta data
\begin{itemize}
    \item Preconditions
    \begin{itemize}
        \item The user must be the author or co-author of the paper
        \item The paper must already be added in the system
    \end{itemize}
    \item Postconditions
    \begin{itemize}
        \item The user is successful in editing the meta data of the paper
    \end{itemize}
    \item Exceptions
    % Still not sure about this
    \begin{itemize}
        %\item An administrator can edit the meta data of any paper in the system
        %\item The head of department can edit the meta data of any paper in the system
        %\item A research leader can edit the meta data of any paper in their research group
    \end{itemize}
\end{itemize}
Research leader being able to view all papers and their progress
\begin{itemize}
    \item Preconditions
    \begin{itemize}
        \item The user must be the research leader of the research group
    \end{itemize}
    \item Postconditions
    \begin{itemize}
        \item The user will be able to view any paper in the research group
    \end{itemize}
    \item Exceptions
    \begin{itemize}
        \item The head of department and administrator can also view the papers in the research group
    \end{itemize}
\end{itemize}
Functionality to back up 
\begin{itemize}
    \item Preconditions
    \item Postconditions
    \item Exceptions
\end{itemize}
Add and remove authors anytime of the paper
\begin{itemize}
    \item Preconditions
    \begin{itemize}
        \item The user must be the primary author of the paper
    \end{itemize}
    \item Postconditions
    \begin{itemize}
        \item The user would be successful in adding or removing authors to or from the paper
    \end{itemize}
\end{itemize}
Show history of papers
\begin{itemize}
    \item Preconditions
    \begin{itemize}
        \item The user can only view the history of their own papers
    \end{itemize}
    \item Postconditions
    \begin{itemize}
        \item The user will successfully view their papers' history
    \end{itemize}
    \item Exceptions
    \begin{itemize}
        \item The head of department can view the history of any paper
        \item The administrator can view the history of any paper
        \item The research leader can view the history of any paper in their research group
    \end{itemize}
\end{itemize}
Staff members being able to access the portal
\begin{itemize}
    \item Preconditions
    \begin{itemize}
        \item Staff members must have profiles on the system
    \end{itemize}
    \item Postconditions
    \begin{itemize}
        \item Staff members successfully access the portal
    \end{itemize}
    \item Exceptions
\end{itemize}
Head Of Department being able to view all papers
\begin{itemize}
    \item Preconditions
    \begin{itemize}
        \item The user type must be the head of department
        \item There can only be one head of department
    \end{itemize}
    \item Postconditions
    \begin{itemize}
        \item The head of department can successfully view all papers
    \end{itemize}
\end{itemize}
The sequence of authors(primary,second etc)
\begin{itemize}
    \item Preconditions
    \begin{itemize}
        \item The sequence of authors must be specified by the user that created the paper
    \end{itemize}
    \item Postconditions
    \begin{itemize}
        \item The sequence of authors is specified
    \end{itemize}
\end{itemize}
log everything 
\begin{itemize}
    \item Preconditions
    \item Postconditions
    \item Exceptions
\end{itemize}
Keep track of units, showing charts to see if they meet the target
\begin{itemize}
    \item Preconditions
    \item Postconditions
    \item Exceptions
\end{itemize}
Count units only when paper has been published
\begin{itemize}
    \item Preconditions
    \begin{itemize}
        \item Units must be assigned to the paper
        \item Paper must already be published
    \end{itemize}
    \item Postconditions
    \begin{itemize}
        \item Units for the paper are counted
    \end{itemize}
\end{itemize}
U.P. is the default of everything
\begin{itemize}
    \item Preconditions
    \begin{itemize}
        \item A user profile must be in the process of being created
    \end{itemize}
    \item Postconditions
    \begin{itemize}
        \item The default institution for every profile will be U.p.
    \end{itemize}
\end{itemize}
Show the intended venue of paper and the type of the paper
\begin{itemize}
    \item Preconditions
    \begin{itemize}
        \item Must be done by a user who is about to create a paper or edited by an author or co-author
    \end{itemize}
    \item Postconditions
    \begin{itemize}
        \item Intended venue and type of paper is shown
    \end{itemize}

\end{itemize}
Send a reminder of when the paper is due
\begin{itemize}
    \item Preconditions
    \begin{itemize}
        \item User to be sent reminder must be an author or co-author of the paper
    \end{itemize}
    \item Postconditions
    \begin{itemize}
        \item Reminder is sent to the user about when the paper is due
    \end{itemize}
\end{itemize}
Unit paper appears by default once it has been stored
\begin{itemize}
    \item Preconditions
    \begin{itemize}
        \item User must specify the units allocated to the paper
    \end{itemize}
    \item Postconditions
    \begin{itemize}
        \item Units for the paper appear by default
    \end{itemize}
\end{itemize}
Search for an author
\begin{itemize}
    \item Preconditions
    \begin{itemize}
        \item The author must be already added into the system
        \item The user searching for the author must be logged in
    \end{itemize}
    \item Postconditions
    \begin{itemize}
        \item The author is found if they exist
    \end{itemize}
\end{itemize}
Head of department being able to view all units per staff
\begin{itemize}
    \item Preconditions
    \begin{itemize}
        \item User must be head of department
    \end{itemize}
    \item Postconditions
    \begin{itemize}
        \item head of department able to view the units allocated to each staff
    \end{itemize}
\end{itemize}
An Administrator having complete access to the portal even on behalf of other users
\begin{itemize}
    \item Preconditions
    \begin{itemize}
        \item The user must be an administrator
        \item The user the administrator is accessing must exist
    \end{itemize}
    \item Postconditions
    \begin{itemize}
        \item The administrator successfully have complete access to the portal
    \end{itemize}
\end{itemize}
Profile of the researcher
\begin{itemize}
    \item Preconditions
    \begin{itemize}
        \item The researcher must be logged in
        \item The researcher must already have a profile on the system
    \end{itemize}
    \item Postconditions
    \begin{itemize}
        \item The researcher can successfull view their profile
    \end{itemize}
\end{itemize}
Can adapt to a change such that it can be extended for other departments
%Is this a use case?
\begin{itemize}
    \item Preconditions
    \item Postconditions
    \item Exceptions
\end{itemize}
A user who's not an author adding a paper for someone else
\begin{itemize}
    \item Preconditions
    \begin{itemize}
        \item The user must be a staff member
    \end{itemize}
    \item Postconditions
    \begin{itemize}
        \item The user successfully creates a paper that another user is an author of
    \end{itemize}
\end{itemize}
%----------------------------------------------------------------------------------------
% To be done by: GP & Hlengekile
%----------------------------------------------------------------------------------------
\subsection{Required functionality}

%----------------------------------------------------------------------------------------
% To be done by: Jedd & JoDan
%----------------------------------------------------------------------------------------
\subsection{Process specifications}

%----------------------------------------------------------------------------------------
% To be done by: Kudzai & Sandile
%----------------------------------------------------------------------------------------
\subsection{Domain Model}

%----------------------------------------------------------------------------------------
% To be done by: Jacques & Jason
%----------------------------------------------------------------------------------------
\section{Open Issues}

Discuss in this section
\begin{itemize}
\item any aspects of the requirements which still need to be specified,
\item around which clarification is still required, as well as
\item any discovered inconsistencies in the requirements.
\end{itemize}


\end{document}
