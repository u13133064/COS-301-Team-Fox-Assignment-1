\documentclass[a4paper,12pt]{article}
\usepackage{blindtext}
\usepackage[utf8]{inputenc}

\begin{document}

\begin{titlepage}

\newcommand{\HRule}{\rule{\linewidth}{0.5mm}} % Defines a new command for the horizontal lines, change thickness here

\center % Center everything on the page
 
%----------------------------------------------------------------------------------------
%	HEADING SECTIONS
%----------------------------------------------------------------------------------------

\textsc{\LARGE University of Pretoria}\\[1.5cm] % Name of your university/college
\textsc{\Large COS 301 - Software Engineering}\\[0.5cm] % Major heading such as course name
\textsc{\large Team Fox}\\[0.5cm] % Minor heading such as course title

%----------------------------------------------------------------------------------------
%	TITLE SECTION
%----------------------------------------------------------------------------------------

\HRule \\[0.4cm]
{ \huge \bfseries Software Requirements Specification and Technology Neutral Process Design}\\[0.4cm] % Title of your document
\HRule \\[1.5cm]
 
%----------------------------------------------------------------------------------------
%	AUTHOR SECTION
%----------------------------------------------------------------------------------------

\begin{minipage}{0.4\textwidth}
\begin{flushleft} \large
\emph{Author(s):}\\
Gian Paolo \textsc{Buffo} % Your name
\end{flushleft}
\end{minipage}
~
\begin{minipage}{0.4\textwidth}
\begin{flushright} \large
\emph{Student number(s):} \\
14446619 % Student number
\end{flushright}
\end{minipage}\\[4cm]


%----------------------------------------------------------------------------------------
%	DATE SECTION
%----------------------------------------------------------------------------------------

{\large \today}\\[3cm] % Date, change the \today to a set date if you want to be precise

 
%----------------------------------------------------------------------------------------

\vfill % Fill the rest of the page with whitespace

\end{titlepage}

\newpage

%=====================================READ ME============================================
%----------------------------------------------------------------------------------------
% I have randomly assigned questions to all of us to get the ball rolling. Obviously,
% anyone can contribute to anything if they feel they have something to add/change.
% The names will be above each section.
% - GP
%----------------------------------------------------------------------------------------


%----------------------------------------------------------------------------------------
% To be done by: Jason
%----------------------------------------------------------------------------------------
\section{Introduction}

The requirements specification should ultimately contain sufficient information such that the
system could be largely developed by a third party without further input. To this end the
requirements must be precise and testable.
The requirements need not be fully specified up-front. One might start with the vision, scope
and architectural requirements, perform an upfront software architecture engineering phase and
then iteratively elicit the detailed requirements for a use case, build, test and deploy the use
case before adding the detailed requirements for the next use case. Such an approach follows
solid engineering phase for the core software infrastructure/architecture with an agile software
development approach within which the application functionality is developed iteratively.

%----------------------------------------------------------------------------------------
% To be done by: GP
%----------------------------------------------------------------------------------------
\section{Vision}

A short discussion of the project vision, i.e. what the client is trying to achieve with the project
and the typical usage scenarios for the outputs of the project.

%----------------------------------------------------------------------------------------
% To be done by: Jacques
%----------------------------------------------------------------------------------------
\section{Background}

A general discussion of what lead to the project including potentially
\begin{itemize}
\item business/research opportunities,
\item opportunities to simplify/improve some aspect of life/work or community,
\item problems your client is currently facing,
\item . . .
\end{itemize}

\section{Architecture Requirements}

The software architecture requirements include the access and integration requirements, quality
requirements and architectural constraints.

%----------------------------------------------------------------------------------------
% To be done by: Hlengekile
%----------------------------------------------------------------------------------------
\subsection{Access Channel Requirements}

Specify the different access channels through which the system's services are to be accessed by
humans and by other systems (e.g. Mobile/Android application clients, Restful web services
clients, Browser clients, . . . ).

%----------------------------------------------------------------------------------------
% To be done by: JoDan
%----------------------------------------------------------------------------------------
\subsection{Quality Requirements}

Specify and quantify each of the quality requirements which are relevant to the system. Examples of quality requirements include performance, reliability, scalability, security, 
flexibility,
maintainability, auditability/monitorability, integrability, cost, usability. Each of these quality
requirements need to be either quantified or at least be specified in a testable way.

%----------------------------------------------------------------------------------------
% To be done by: Jason
%----------------------------------------------------------------------------------------
\subsection{Integration Requirements}

This section specifies any integration requirements for any external systems. This may include
\begin{itemize}
\item the integration channel to be used,
\item the protocols to be used,
\item API specifications in the form of UML interfaces and/or technology-specific API specifications (e.g. WSDLs, CORBA IDLs, . . . ), and
\item any quality requirements for the integration itself (performance, scalability, reliability, security, auditability, . . . ).
\end{itemize}

%----------------------------------------------------------------------------------------
% To be done by: Jedd
%----------------------------------------------------------------------------------------
\subsection{Architecture Constraints}

This specifies any constraints the client may specify on the system architecture include
\begin{itemize}
\item technologies which MUST be used,
\item architectural patterns/frameworks which must be used (e.g. layering, Services Oriented Architectures, . . . )
\item . . .
\end{itemize}

\section{Functional requirements and application design}
This section discusses the application functionality required by users (and other stakeholders).\\
\textbf{Additional information for this section is provided in the spec uploaded to the CS website.}

%----------------------------------------------------------------------------------------
% To be done by: Sandile
%----------------------------------------------------------------------------------------
\subsection{Use case prioritization}
Critical
\begin{itemize}
\item Adding a conference paper
\item Adding a author to a conference paper
\item User being able to see all papers they have added or mentioned as authors to
\item Adding a researcher to a research group
\item Editing meta data 
\item Research leader being able to view all papers and their progress
\item The state of the paper (submitted, waiting, rejected, published)
\item Functionality to back up 
\item add and remove authors anytime of the paper
\item Show history of papers
\item Stuff members being able to access the portal

\end{itemize}
Important
\begin{itemize}
\item Head Of Department being able to view all papers
\item The sequence of authors(primary,second etc)
\item log everything 
\item Keep track of units, showing charts to see if they meet the target
\item Count units only when paper has been published
\item U.P is the default of everything
\item Show the intend venue of paper and the type of the paper
\item Send a reminder of when the paper is due
\item unit paper appears by default once it has been stored
\item Search for an author
\item Head of department being able to view all units per stuff
\item An Administrator having complete access to the portal even on behalf of other users
\item Head of Department having complete access to the portal even on behalf of other users
\end{itemize}
Nice to have
\begin{itemize}
 \item profile of the researcher
 \item can adapt to a change such that it can be extended for other departments
 \item A user who's not an author adding a paper for someone else
\end{itemize}

%----------------------------------------------------------------------------------------
% To be done by: Kudzai
%----------------------------------------------------------------------------------------
\subsection{Use case/Services contracts}

%----------------------------------------------------------------------------------------
% To be done by: GP & Hlengekile
%----------------------------------------------------------------------------------------
\subsection{Required functionality}

%----------------------------------------------------------------------------------------
% To be done by: Jedd & JoDan
%----------------------------------------------------------------------------------------
\subsection{Process specifications}

%----------------------------------------------------------------------------------------
% To be done by: Kudzai & Sandile
%----------------------------------------------------------------------------------------
\subsection{Domain Model}

%----------------------------------------------------------------------------------------
% To be done by: Jacques & Jason
%----------------------------------------------------------------------------------------
\section{Open Issues}

Discuss in this section
\begin{itemize}
\item any aspects of the requirements which still need to be specified,
\item around which clarification is still required, as well as
\item any discovered inconsistencies in the requirements.
\end{itemize}


\end{document}