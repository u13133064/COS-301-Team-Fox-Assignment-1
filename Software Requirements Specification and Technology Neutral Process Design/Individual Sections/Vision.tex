\documentclass[a4paper,12pt]{article}
\usepackage{blindtext}
\usepackage[utf8]{inputenc}

% A short discussion of the project vision, i.e. what the client is trying to achieve with  % the project and the typical usage scenarios for the outputs of the project.

\begin{document}

\section{Vision}

With the implementation of this system, the client is trying to create a central piece of software that can be used by all staff members of the Computer Science department of the University of Pretoria to maintain their academic publications.

However, the software will provide much greater functionality than merely listing all publications. One of the main features of the system will be the ability to add and edit publications, as well as specify a multitude of metadata items such as the title, (co)-authors, deadlines, progress towards completion, status (published, accepted, submitted etc.) and intended venue (conference, journal).

Additionally, the system will be used to provide users (who are authors of publications) with all manner of information regarding their Research Output Units and funding. This includes information such as expected units and funding, obtained units and funding, and shortfall of units.

The system will provide outputs in the form of an Excel spreadsheet which will illustrate the aforementioned author details both in a tabular fashion and graphically, most likely in the form of bar and/or line graphs. A sheet will be generated for each individual author, and a master sheet will be generated for the Research Leaders and Head of Department respectively.

A typical usage scenario of the system will be as follows:
\begin{itemize}

\item A user, who in this case is an author as well as the Head of Department (giving him/her administrator rights), logs in to the system.
\item The user then adds a new publication on his/her profile page, filling in the relevant metadata items. The user will by default be added as an author.
\item The user realises that he/she made an error while filling in the title, and edits the title appropriately.
\item The units corresponding to the publication's venue (which have been assigned by the system) are automatically added to the user, and all unit-related calculations are made
\item The user's profile page is updated with the new publication and all related information.
\item The user then chooses to generate the Excel document which contains not only information about his/her Research Output Units and funding in a tabular and graphical fashion, but also similar information for all other members of the department, who are users. This is because the user has Head of Department rights. 

\end{itemize}

\end{document}